%---------------------------------------------------------------------------%
%->> Backmatter
%---------------------------------------------------------------------------%
\chapter[致谢]{致\quad 谢}\chaptermark{致\quad 谢}% syntax: \chapter[目录]{标题}\chaptermark{页眉}
三光日月星, 十年本硕博。

感谢何国威老师、杨晓雷老师、张鑫磊老师、吕钰老师、和徐多老师。
120这个数字很有意思,这是何老师课题组博后们的办公室门号,又恰好是急救电话号码。
有那么一段时间,238C遇到困难时会不约而同地想: ``快, 打120"。
120办公室有非常多猛人,被称为CFD仙人的刘毅师兄,做湍流理论青出于蓝而胜于蓝的吴霆师兄,本科就发JFM的刘昊辰师兄,网格师傅王春雨师兄,以及写文章非常厉害的张鑫磊师兄, ......, 可惜我做的课题方向非常有限,没有和120所有的人打完交道。
我们给何老师做报告一般也在120会议室,事实上感觉那里更像手术台,进去后就等着被开刀吧,不打麻药的那种。

感谢我的同学(刘晓豪、陈丹阳、臧振宇,谢卓宇, 陈与,吴凡、还有王嘉麟 ),
感谢我那些个性鲜明的师弟:不知疲倦的、数学物理功底最深的王笼同学;什么都会什么都懂还出活特别快的罗清勇同学;基础特别扎实、成果特别亮眼的张风顺同学;极其聪敏、消息灵通得像信号基站的程傲同学;
特别鸣谢来自中科大的王友华师弟,在我接手他的预解算子工作时,他给了我毫无保留和细致入微的帮助,帮我顺利快速地掌握了预解算子的核心思想。

还有在课题组匆匆而过的刘伯峰、陈辂师弟,张慧颖师姐。

还要感谢许昭越师兄,周志腾师兄,李世隆师兄。


\chapter{作者简历及攻读学位期间发表的学术论文与研究成果}

%\textbf{本科生无需此部分}。

\section*{作者简历:}
2016年9月--2020年6月, 华中科技大学土木工程与力学学院, 工程力学系, 学士

2020年9月--2026年6月, 中国科学院力学研究所, 流体力学, 博士

\section*{攻读学位期间的学术论文发表情况:}
 {
  \setlist[enumerate]{}% restore default behavior
  \begin{enumerate}[nosep]
      \item \textbf{Wu Chutian}, Zhang Xin-Lei, He Guowei, 2025. Neural operator-based stochastic forcing for resolvent prediction of space-time turbulence statistics in channel flows. \textbf{\textit{Journal of Fluid Mechanics}} 1024, A1. %\href{https://doi.org/10.1017/jfm.2025.10847}{https://doi.org/10.1017/jfm.2025.10847}
      \item \textbf{Wu Chutian}, Zhang Xin-Lei, Xu Duo, He Guowei, 2025. A framework for learning symbolic turbulence models from indirect observation data via neural networks and feature importance analysis. \textbf{\textit{Journal of Computational Physics}} 114068. %\href{https://doi.org/10.1016/j.jcp.2025.114068}{https://doi.org/10.1016/j.jcp.2025.114068}
      \item \textbf{Wu Chutian}, Wang Shizhao, Zhang Xin-Lei, He Guowei, 2023. Explainability analysis of neural network-based turbulence modeling for transonic axial compressor rotor flows. \textbf{\textit{Aerospace Science and Technology}} 141, 108542. %\href{https://doi.org/10.1016/j.ast.2023.108542}{https://doi.org/10.1016/j.ast.2023.108542}
      \item \textbf{Wu Chutian}, Yang Xiaolei, Zhu Yaxin, 2021. On the design of potential turbine positions for physics-informed optimization of wind farm layout. \textbf{\textit{Renewable Energy}} 164, 1108-1120. %\href{https://doi.org/10.1016/j.renene.2020.10.060}{https://doi.org/10.1016/j.renene.2020.10.060}
      \item Dong Guodan, Qin Jianhua, \textbf{Wu Chutian}, Xu Chang, Yang Xiaolei, 2026. Reinforcement learning-enhanced genetic algorithm for wind farm layout optimization. \textbf{\textit{Renewable Energy}} 259, 125093. %\href{https://doi.org/10.1016/j.renene.2025.125093}{https://doi.org/10.1016/j.renene.2025.125093}
  \end{enumerate}
 }


\section*{攻读学位期间的会议报告:}
 {
  \setlist[enumerate]{}% restore default behavior
  \begin{enumerate}[nosep]
      \item \textbf{吴楚畋}, 杨晓雷, 一种基于物理的风机布置优化方法, 第十二届全国流体力学大会, 深圳, 2020年11月
      \item \textbf{吴楚畋}, 张鑫磊, 跨音速轴流压气机转子绕流的神经网络湍流建模及其可解释性分析, 第十三届全国流体力学大会, 哈尔滨, 2024年8月
      \item \textbf{吴楚畋}, 张鑫磊, 神经网络湍流模型的解释性分析方法: 结合特征重要性和符号回归, 中国力学大会, 长沙, 2025年7月
  \end{enumerate}
 }
\section*{奖励荣誉}
\begin{itemize}
    \item 中国科学院大学三好学生, 2020
    \item 研究生国家奖学金, 2025
\end{itemize}


\cleardoublepage[plain]% 让文档总是结束于偶数页, 可根据需要设定页眉页脚样式, 如 [noheaderstyle]
%---------------------------------------------------------------------------%
