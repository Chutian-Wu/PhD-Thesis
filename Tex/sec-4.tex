
\chapter{基于神经算子的预解分析色噪声随机力建模}
\section{引言}
湍流中大尺度结构的时空特性对于涉及流动诱发噪声和结构振动的工程应用具有实际意义~\citep{williams1978noise,blake2017mechanics}。
预测时空流动统计量通常依赖于尺度解析模拟, 例如大涡模拟~\citep{he2004computation,he2017space} 和直接数值模拟~\citep{choi1990space}。
然而, 此类方法需要解析大多数流动尺度以捕捉尺度间的能量传递, 对于高Reynolds数流动而言, 由于显著的尺度分离, 其计算成本令人望而却步。
因此, 开发一种高效预测大尺度湍流时空特性的方法具有迫切需求。

基于预解(resolvent)的方法~\citep{Hwang_Cossu_2010b, McKEON_SHARMA_2010, McKeon_2017} 通过将Fourier变换后的Navier-Stokes方程视为一个随机力的线性系统, 能够高效地刻画大尺度湍流结构的时空特性。
具体而言, 该方法将非线性对流项视为作用于平均流线性化Navier-Stokes方程上的随机力。
由此, 湍流动力学可被分析为一个输入-输出系统。
特定时空尺度的随机输入激励可通过线性预解算子产生相应的速度响应。
基于预解的方法已成为预测湍流时空统计量的强大工具, 实现了低阶建模~\citep{taira2017modal,li2024resolvent}、状态估计~\citep{Towne_Lozano_Duran_Yang_2020,ying2023resolvent}、流动分析~\citep{symon2018non} 与控制~\citep{luhar2014opposition,yeh2019resolvent}。

基于预解方法的预测精度高度依赖于随机力的建模。
通常使用时域白噪声且空间解相关的强迫作为该线性模型的输入激励~\citep{Jovanovic_2001}。
然而, 非线性强迫项本质上是``有色"的, 即时域和空间上均相关~\citep{Morra_Henningson_2021}。
白噪声强迫会导致速度谱的显著偏差, 这已在文献中得到广泛认知~\citep{Zare_Georgiou_2017, Morra_Cossu_2019, Nogueira_Morra_Martini_Cavalieri_Henningson_2021, wang2025significant}。
特别地, 时域白噪声强迫无法复现正确的速度协方差, 因为它导致Lyapunov方程右侧矩阵具有不定的正定性~\citep{Zare_Georgiou_2017}。
此外, 时域白噪声强迫产生无限的频率谱带宽, 导致Taylor时间微尺度消失, 进而造成时间相关性预测不准确~\citep{Wu_2021_PRF, Wu_He_2023}。
因此, 开发有色的非线性强迫模型对于提高基于预解方法的预测精度至关重要。

已有多种方法通过考虑强迫的"颜色"来改进基于预解方法的非线性强迫模型。
例如, \citet{Zare_Georgiou_2017} 为线性Navier-Stokes方程开发了一种有色随机力模型, 该模型采用最大熵公式并结合核范数正则化进行秩最小化。
在给定单点速度相关性和平均速度的条件下, 其方法能准确预测两点速度相关性。
此外, \citet{Wu_2021_PRF, Wu_He_2023} 提出了一种时空相干随机力模型, 该模型在复合预解框架中引入了涡旋阻尼以考虑随机扫掠效应。
在给定扫掠速度和涡旋阻尼尺度的情况下, 该方法能够捕捉通道流动中的特征去相关时间尺度。
总体而言, 上述工作能够基于现有数据有效近似非线性强迫。

基于涡黏性的方法为部分模拟强迫颜色提供了另一途径~\citep{del_Alamo_Jimenez_2006, Hwang_Cossu_2010a, Hwang_Cossu_2010b}, 而无需依赖现有速度数据。
具体而言, 该方法将非线性强迫分解为涡黏性项(旨在表征从大尺度向小尺度的能量传递~\citep{Symon_2023})和白噪声随机力项(代表非耗散能量成分) 。
在预解分析中加入涡黏性已被证明能显著提高速度脉动时空能谱的预测精度~\citep{Morra_Cossu_2019, Holford_Lee_Hwang_2023}。
这种改进可归因于, 涡黏性强化的强迫投射到预解算子线性次优模态上的方式, 与DNS数据所呈现的相似, 这表明涡黏性项部分捕捉了非线性强迫的时空相干性~\citep{Morra_Henningson_2021}。
基于涡黏性强化的预解方法已被证明在多种流动应用中具有价值, 例如基于肋条的减阻~\citep{ran2021model}、射流噪声建模~\citep{pickering2021resolvent}、超声速空腔流动的主动控制~\citep{liu2021unsteady} 等。

尽管基于涡黏性强化的预解方法被广泛应用, 但该方法需要预先已知涡黏性和平均速度。
对于湍流通道流动, 通常使用Cess模型~\citep{Cess_1958, Reynolds_Tiederman_1967} 来提供涡黏性剖面, 并通过在Boussinesq假设下求解Reynolds平均Navier-Stokes (RANS)方程进一步获得平均速度。
然而, Cess模型仅适用于通道和管道流动, 对于包括射流和翼型流动在内的通用应用~\citep{von2024role}, 则需借助常用的湍流模型, 例如Spalart-Allmaras模型~\citep{spalart1992} 和 $k$-$\omega$ 模型~\citep{wilcox1998turbulence}。
此外, 这些涡黏模型在表征RANS方程中的Reynolds应力时存在固有不确定性~\citep{Duraisamy2019annurev}, 使其难以在不同流动场景中良好泛化。
RANS提供的涡黏性中的此类不确定性也会显著影响涡黏性强化预解方法在速度谱预测中的精度。
另一方面, 由于采用时域白噪声随机激励, 涡黏性强化的预解方法仍无法完全复现速度统计量, 特别是在时间相关性方面~\citep{Wu_He_2023}。
因此, 非线性强迫模型值得进一步发展, 以提升结合RANS提供涡黏性的预解方法的预测性能。

已有研究探索了数据驱动方法来改进非线性强迫模型, 即基于高保真数据优化涡黏性剖面或白噪声随机力的幅值。
例如, \citet{pickering2021resolvent} 通过将主导预解模态与大涡模拟数据的谱本征正交分解(SPOD)模态对齐, 确定了最优涡黏性。
这能在涡黏性强化预解框架下实现速度响应的最优预测。
此外, 优化后的涡黏性可改进从非线性强迫到其线性响应的传递函数, 以用于流动控制应用。
另一种方法是, \citet{Holford_Lee_Hwang_2023} 基于DNS数据优化了方向相关的白噪声强迫幅值, 从而改进了湍流Reynolds应力的预测。
这些方法可产生最优的非线性强迫场, 但它们是针对特定流动构型定制的, 限制了其向其他流动条件和RANS模型的泛化能力。

基于神经网络的函数近似已在开发与Reynolds应力~\citep{Ling2015,zhang2022} 和亚格子应力~\citep{zhou2019subgrid,park2021toward} 相关的可泛化湍流模型中展现出显著潜力。
特别是, 神经算子~\citep{kovachki2023neural} 能够从高保真数据中学习无限函数空间之间的映射, 从而实现跨不同模型输入的泛化。
深度算子网络(DeepONet) ~\citep{lu2021learning} 是一种广泛使用的神经算子, 它由两个子网络组成: 一个分支网络用于编码输入函数, 一个主干网络用于编码待求输出函数的定义域位置。
该架构已成功应用于多种流动问题, 例如学习湍流燃烧封闭模型~\citep{taassob2024pinn}、构建气泡动力学的代理模型~\citep{lin2021seamless}、预测高速边界层中的不稳定性波~\citep{di2023neural}、加速圆锥体上过渡流的数据同化~\citep{morra_ml_2024}, 以及模拟高超音速反应流动的多物理场~\citep{mao2021deepm,cai2021deepm}。
此外, DeepONet能够灵活地扩展输出函数的定义域~\citep{Lu_2022}。
鉴于这些优势, 基于DeepONet的神经算子有望构建数据驱动的非线性强迫模型, 从而提高基于预解方法在不同RANS模型和流动条件下的泛化能力。

在本研究中, 我们提出了基于神经算子的非线性强迫模型, 用于预测湍流通道流动中大尺度结构的时空统计量。
该方法整合了涡黏性强化预解算子和基于神经算子的随机力模型。
具体而言, 我们构建了一个基于DeepONet的模型, 以根据DNS数据将涡黏性和平均速度映射到随机力谱。
该随机力被认为是方向相关的、在壁面法向非均匀分布, 并在空间和时间上均相关。
进一步地, 所得的随机力可通过涡黏性强化的预解算子生成时空速度统计量。
通过这种方式, 所提出的模型能够产生时域"有色"的随机力, 从而实现对大尺度流动结构时空统计量的准确预测。
特别地, 神经算子强化的预解模型允许基于RANS提供的平均速度和涡黏性来预测速度谱, 而无需依赖现有数据。
因此, 学习到的模型可适用于相似的流动条件, 例如不同Reynolds数的情况。
此外, 基于DeepONet的强迫模型能够很好地泛化到不同的涡黏性剖面及相应的平均速度, 为速度统计量提供相对一致的预测。

本文其余部分结构如下。
首先, 第~\ref{sec: problem} 节阐述了基于预解方法的随机力建模问题。
随后, 第~\ref{sec: method} 节介绍了所提出的预解神经算子。
第~\ref{sec: data} 节描述了用于训练神经算子的数据生成过程。
进一步地, 第~\ref{sec: results} 节评估并讨论了学习到的随机力模型在强迫统计量、速度统计量以及模型泛化能力方面的预测性能。
最后, 第~\ref{sec: conclusion} 节对全文进行总结。

\section{预解分析与随机力建模}
\label{sec: problem}

\subsection{预解算子}
针对湍流通道流动, 介绍基于预解方法的数学描述。
基于涡黏性模拟湍流Reynolds应力的不可压湍流通道流动线性化Navier-Stokes方程为~\citep{Cossu_Pujals_Depardon_2009}:
\begin{subequations}
    \label{equ: NS-spatial}
    \begin{align}
        \nabla\cdot\bm{u}                                                   & =0, \label{equ:div-free}                                                                                   \\
        \p_t\bm{u}+\bm{U}\cdot\nabla\bm{u}+\bm{u}\cdot\nabla\bm{U}+\nabla p & =\nabla\cdot\sbra{\bra{\nu+\nu_t}\bra{\nabla\bm{u}+\nabla\bm{u}^\top}}+\bm{f} \text{,} \label{equ: NS-vec}
    \end{align}
\end{subequations}
其中 $\bm{u}=\sbra{u(\bm{x},t),v(\bm{x},t),w(\bm{x},t)}^\top$ 表示脉动速度场。
计算域采用笛卡尔坐标系 $\bm{x}=(x,y,z)^\top$ 表示, 分别对应流向($x$)、壁面法向($y$)和展向($z$)。
平均速度 $\bm{U}=\sbra{U(y),0,0}^\top$ 仅有流向分量, 其在壁面法向 $y\in\sbra{-h,h}$ 上变化, 其中 $h$ 为通道半高度。
变量 $p$、$\nu$ 和 $\nu_t$ 分别表示脉动压力、运动粘性系数和涡黏性系数。
随机力 $\bm{f}$ 用于描述涡黏性项与精确非线性项之间的残差~\citep{Holford_Lee_Hwang_2023}, 其定义为:
\begin{equation}\label{equ: stochastic-forcing-defi}
    \bm{f}\equiv-\nabla\cdot\bra{\bm{uu}-\abra{\bm{uu}}}-\nabla\cdot\sbra{\nu_t\bra{\nabla\bm{u}+\nabla\bm{u}^\top}} \text{,}
\end{equation}
其中 $\abra{\cdot}$ 表示系综平均。

由于通道流动在壁面平行($x$--$z$)平面内的均匀性, 方程~\eqref{equ: NS-spatial} 可在该方向进行Fourier变换。
进一步地, 应用标准的压力消去步骤~\citep{Schmid_2002}, 得到演化方程为:
\begin{equation}
    \label{equ: dynamic-sys}
    \frac{\p}{\p t}\hat{\bm{q}}(\bm{k},y,t)=\bm{A}_{\nu_t}\hat{\bm{q}}(\bm{k},y,t)+\bm{B}\hat{\bm{f}}(\bm{k},y,t) \text{,}
\end{equation}
其中 $\hat{\bm{q}}=\sbra{\hat{v},\hat{\omega}_y}$ 是状态向量, 包含壁面法向速度和壁面法向涡量的Fourier模态。
波数向量由 $\bm{k}=\bra{k_x, k_z}$ 给出, 其中 $k_x$ 和 $k_z$ 分别表示流向波数和展向波数。
速度向量与状态向量通过 $\hat{\bm{u}}=\bm{C}\hat{\bm{q}}$ 和 $\hat{\bm{q}}=\bm{D}\hat{\bm{u}}$ 相关联, 其中 $\bm{C}$ 和 $\bm{D}$ 为变换矩阵。
进一步考虑到流动在时间上的统计平稳性, 方程~\eqref{equ: dynamic-sys} 中的动态系统可在时间维度进行Fourier变换, 得到速度对随机力的响应为:
\begin{equation}
    \label{equ: resolvent}
    \tilde{\bm{u}}\bra{\bm{k},\omega,y}=\bm{R}\tilde{\bm{f}}\bra{\bm{k},\omega,y} \text{,}
\end{equation}
其中 $\omega$ 为频率, $\tilde{\bm{u}}$ 和 $\tilde{\bm{f}}$ 分别为速度向量和随机力向量的Fourier模态, $\bm{R}$ 为预解算子, 其定义为:
\begin{equation}
    \label{equ: resolvent-operator}
    \bm{R}=-\bm{C}\bra{\ii\omega\m{I}+\bm{A}_{\nu_t}}^{-1}\bm{B} \text{.}
\end{equation}
在上述公式中, $\ii$ 为虚数单位, $\m{I}$ 为单位矩阵。
对于每个波数和频率, 方程~\eqref{equ: resolvent} 在壁面法向使用 $N_y$ 个切比雪夫配置点进行离散。
边界条件在两侧壁面施加为 $\tilde{v}(\pm h)=\p_y\tilde{v}(\pm h)=\tilde{\omega}_y\bra{\pm h}=0$。
向量 $\tilde{\bm{u}}$ 和 $\tilde{\bm{f}}$ 的维度均为 $3N_y\times 1$, 分别包含速度分量和随机力分量的Fourier模态。
预解算子 $\bm{R}$ 融合了涡黏性剖面 $\nu_t(y)$ 和平均流剖面 $U(y)$, 这两者通过使用基于涡黏性的湍流模型求解Reynolds平均Navier-Stokes (RANS)方程获得~\citep{Pope_2000}。
在槽道流动中, 有多种湍流模型可用于此目的, 包括Cess模型~\citep{Cess_1958, Reynolds_Tiederman_1967}、Spalart-Allmaras模型~\citep{spalart1992} 和 $k$-$\omega$ 模型~\citep{wilcox1998turbulence,Wilcox_2008}, 简要说明见附录~\ref{app: RANS}。


矩阵 $\bm{A}_{\nu_t},\bm{B},\bm{C}$ 和 $\bm{D}$ 的详细表达式如下。
The operators in the evolution equation Eq.~\eqref{equ: dynamic-sys} are defined as
\begin{equation}
    \bm{A}_{\nu_t}=\begin{bmatrix}
        \Delta^{-1}\mathcal{L}_\text{OS} & 0                     \\
        -\ii k_z U'                      & \mathcal{L}_\text{SQ}
    \end{bmatrix},\quad
    \bm{B}=\begin{bmatrix}
        -\ii k_x\Delta^{-1}{\p_y} & -k^2\Delta^{-1} & -\ii k_z\Delta^{-1}{\p_y} \\
        \ii k_z                   & 0               & -\ii k_x
    \end{bmatrix},
\end{equation}
where $\mathcal{L}_\text{OS}$ is the generalized Orr-Sommerfeld and $\mathcal{L}_\text{SQ}$ is the Squire operator~\citep{Cossu_Pujals_Depardon_2009}, which are defined as
\begin{subequations}
    \begin{align}
        \mathcal{L}_\text{OS} & =-\ii k_x\bra{U\Delta-U''}+\bra{\nu+\nu_t}\Delta^2+2\nu'_t\Delta\p_y+\nu''_t\bra{\p_{yy}+k^2}, \\
        \mathcal{L}_\text{SQ} & =-\ii k_x U+\bra{\nu+\nu_t}\Delta+\nu'_t\p_y.
    \end{align}
\end{subequations}
In the operators above, the prime notation denotes differentiation with respect to the wall-normal coordinate $y$. Additionally, $k^2=k_x^2+k_z^2$ and the Laplacian in wavenumber space is $\Delta=\p_{yy}-k^2$.
All operators are discretised via a spectral collocation method with Chebyshev-based differentiation matrices.
The transformation matrices relating velocity vector $\tilde{\bm{u}}$ and state vector $\tilde{\bm{q}}$ are
\begin{equation}
    \bm{C}=\frac{1}{k^2}\begin{bmatrix}
        \ii k_x\p_y  & -\ii k_z \\
        k^2          & 0        \\
        \ii k_z \p_y & \ii k_x
    \end{bmatrix},\quad
    \bm{D}=\begin{bmatrix}
        0       & 1 & 0        \\
        \ii k_z & 0 & -\ii k_x
    \end{bmatrix}.
\end{equation}


在给定涡黏性和平均速度的情况下, 可根据方程~\eqref{equ: resolvent-operator} 得到预解算子 $\bm{R}$。
进一步地, 在给定一个建模的随机力时, 可由方程~\eqref{equ: resolvent} 推导出速度交叉谱为:
\begin{equation}
    \label{equ: RPR}
    \bm{S}=\bm{R}\bm{P}\bm{R}^\herm \text{,}
\end{equation}
其中 $\bm{S}$ 和 $\bm{P}$ 分别为速度与随机力的交叉谱。
它们可表述为:
\begin{equation}
    \begin{aligned}
        \bm{S}(k_x,k_z,\omega,y,y') & =\langle{\tilde{\bm{u}}(k_x,k_z,\omega,y)\tilde{\bm{u}}^\herm(k_x,k_z,\omega,y')}\rangle,         \\
        \bm{P}(k_x,k_z,\omega,y,y') & =\langle{\tilde{\bm{f}}(k_x,k_z,\omega,y)\tilde{\bm{f}}^\herm(k_x,k_z,\omega,y')}\rangle \text{,}
    \end{aligned}
\end{equation}
其中上标 $\herm$ 表示共轭转置(Hermitian转置) 。
为记法简便, 下文将省略对波数和频率的依赖关系。
Hermitian 矩阵 $\bm{S}$ 和 $\bm{P}$ 均为块结构矩阵, 具体形式如下:
\begin{equation}
    \bm{S}=\begin{bmatrix}
        \abra{\tilde{u}\tilde{u}^\herm} & \abra{\tilde{u}\tilde{v}^\herm} & \abra{\tilde{u}\tilde{w}^\herm} \\
        \abra{\tilde{v}\tilde{u}^\herm} & \abra{\tilde{v}\tilde{v}^\herm} & \abra{\tilde{v}\tilde{w}^\herm} \\
        \abra{\tilde{w}\tilde{u}^\herm} & \abra{\tilde{w}\tilde{v}^\herm} & \abra{\tilde{w}\tilde{w}^\herm} \\
    \end{bmatrix},\quad
    \bm{P}=\begin{bmatrix}
        \langle\tilde{f}_x\tilde{f}_x^\herm\rangle & \langle\tilde{f}_x\tilde{f}_y^\herm\rangle & \langle\tilde{f}_x\tilde{f}_z^\herm\rangle \\
        \langle\tilde{f}_y\tilde{f}_x^\herm\rangle & \langle\tilde{f}_y\tilde{f}_y^\herm\rangle & \langle\tilde{f}_y\tilde{f}_z^\herm\rangle \\
        \langle\tilde{f}_z\tilde{f}_x^\herm\rangle & \langle\tilde{f}_z\tilde{f}_y^\herm\rangle & \langle\tilde{f}_z\tilde{f}_z^\herm\rangle
    \end{bmatrix}.
\end{equation}
块矩阵 $\bm{S}$ 和 $\bm{P}$ 由 $3\times 3$ 个子矩阵 $S_{ij}=\abra{\tilde{u}_i\tilde{u}_j^\herm}$ 和 $P_{ij}=\langle\tilde{f}_i\tilde{f}_j^\herm\rangle$ 构成(其中 $i,j=1,2,3$) , 每个子矩阵的尺寸为 $N_y\times N_y$。
在每个子矩阵内部, 对角元素表示相应向量分量的单点相关性, 而非对角元素则描述壁面法向的相关性。
总而言之, 速度的交叉谱 $\bm{S}$ 可通过预解算子由强迫谱 $\bm{P}$ 得到。



\subsection{随机力谱}
\label{sec: Modeling of stochastic forcing spectra}
为对随机力进行建模, 规定其统计结构至关重要。
方程~\eqref{equ: stochastic-forcing-defi} 中的随机力 $\bm{f}$ 已知是时域有色、尺度相关、在壁面法向空间相关、其各分量之间存在互相关, 且强度沿槽道高度非均匀分布~\citep{Holford_Lee_Hwang_2023}。
遵循先前研究~\citep{Morra_Cossu_2019, Morra_Henningson_2021, Gupta_2021, Holford_Lee_Hwang_2023, Holford_Lee_Hwang_2024}, 本文假设随机力在空间上解相关且不存在分量间的互相关。
通过这种方式, 待建模的强迫谱 $\bm{P}$ 从一个完整的 $3N_y\times 3N_y$ 矩阵简化为其对角线元素, 尺寸为 $3N_y$。

在本研究中, 我们将随机力规定为时域有色且尺度相关, 其强度随壁面法向位置变化, 同时满足动能守恒。
具体而言, 目标随机力谱被表述为:
\begin{equation}
    \label{equ: P}
    \bm{P}(\bm{k},\omega,y,y')=\gamma(\bm{k},\omega)\bm{W}(\bm{k},\omega,y)\delta\bra{y-y'} \text{,}
\end{equation}
其中 $\bm{P}$ 是一个对角矩阵, 它排除了壁面法向相关性和分量间的互相关性, 仅保留了指定波数和频率下的单点自相关。
矩阵 $\bm{W}$ 表示壁面法向的相对强度分布, 可以从随机力的无散部分的单点自相关中获取。
系数 $\gamma$ 用于缩放强迫强度, 以匹配DNS预测的动能。
系数 $\gamma$ 和矩阵 $\bm{W}$ 被确定为:
\begin{equation}
    \label{equ: target}
    \gamma\bra{\bm{k},\omega}=\frac{\| \bm{S}^\text{(DNS)}\|_*}{\| \bm{R}\bm{W}\bm{R}^\herm \|_*},\quad \bm{W}=\bra{\bm{L}\bm{S}^\text{(DNS)}\bm{L}^\herm}\circ\m{I} \text{.}
\end{equation}
在上式中, $\circ$ 表示逐元素矩阵积(Hadamard积) , 而 $\|\cdot \|_*$ 为核范数~\citep{Bhatia1997}, 其定义为:
\begin{equation}
    \label{equ: norm-y}
    \|\bm{S}\|_*= \sum_{i=1}^{3N_y}\sigma_i(\bm{S}),
\end{equation}
它由所有奇异值 $\sigma_i(\bm{S})$ 的总和给出。
由于奇异向量的正交性, 速度交叉谱矩阵 $\bm{S}$ 的核范数 $\|\bm{S}\|_*$ 代表了在给定波数和频率下沿壁面法向积分的湍动能的两倍~\citep{Jovanovic_Bamieh_2005}, 即:
\begin{equation}
    \|\bm{S}\|_* = \iint_{-h}^{h}\sum_{i=1}^{3}\abra{\tilde{u}_i(y)\tilde{u}_i^\herm (y')}\delta(y-y')\dif{y}\dif{y'}.
\end{equation}
该公式用于实际计算核范数。
式~\eqref{equ: target} 中的算子 $\bm{L}$ 被定义为~\citep{Morra_Cossu_2019}:
\begin{equation}
    \bm{L}=-\bm{C}\bra{\ii\omega\m{I}+\bm{A}_{\nu_t}}\bm{D},
\end{equation}
该算子用于重构随机力的无散部分, 即:
\begin{equation}\label{equ: fs}
    \bm{L}\tilde{\bm{u}}=\tilde{\bm{f}}^\text{s}.
\end{equation}
需注意, 随机力可分解为无散部分和有势部分, 即:
\begin{equation}
    \tilde{\bm{f}}=\tilde{\bm{f}}^\text{s}+\tilde{\bm{f}}^\text{r},
\end{equation}
其中无散部分满足散度为零(即 $\nabla\cdot\tilde{\bm{f}}^\text{s}=0$) , 而有势部分满足旋度为零(即 $\nabla\times\tilde{\bm{f}}^\text{r}=0$) 。
由于有势强迫 $\bm{f}^\text{r}$ 位于算子 $\bm{B}$ 的零空间中~\citep{Rosenberg_2019}, 仅随机力的无散分量可从速度中重构。
综上所述, 本工作的目标是建模式~\eqref{equ: P} 中定义的随机力的单点自相关 $\bm{P}$。

% \section{方法}
% \label{sec: method}

\subsection{用于随机力建模的神经算子}
引入神经算子来建模随机力分量的频率谱。
基于方程~\eqref{equ: fs}, 若给定来自DNS的精确脉动速度响应 $\tilde{\bm{u}}$, 随机力则与平均速度和涡黏性相关。
鉴于此, 我们假设随机力是平均速度和涡黏性的函数, 并可采用神经算子方法来近似。

采用基于DeepONet的神经算子来表示随机力谱。
神经算子具有一个吸引人的特性: 其输入与输出函数无需共享相同的定义域, 从而能够实现任意函数空间之间的映射~\citep{Lu_2022}。
利用这种灵活性, 我们构建了一个神经算子, 其中输入函数定义在无量纲壁面距离 $1-\abs{y}/h$ 上, 而输出函数则扩展到波数-频率-距离域 $(k_xh,k_zh,\omega h/U_\text{m},1-\abs{y}/h)$。
具体而言, 输入包含两个剖面, 即沿壁面法向的平均速度和涡黏性; 输出则包含三个频率谱, 分别对应随机力的三个分量。
遵循~\citet{Lu_2022} 的工作, 我们采用了能够处理多个输入和输出函数的扩展版DeepONet。
如图~\ref{fig: neural-operator-schem} 所示, 该神经网络由分支网络和主干网络构成。
分支网络的输入层分为两组, 分别编码平均速度剖面和涡黏性剖面。
主干网络的输入为一组用于评估强迫谱输出函数的位置点, 包括流向和展向波数 $(k_x, k_z)$、时间频率 $\omega$, 以及壁面法向距离 $1-|y|/h$。
分支网络的输出层分为三组 $\m{b}_i=\sbra{\mathrm{b}_{i1},\mathrm{b}_{i2},\cdots,\mathrm{b}_{in}}^\top$ (其中 $i=1,2,3$), 每组对应随机力的一个分量, 且其维度与主干网络的输出 $\m{a} =\sbra{\mathrm{a}_1,\mathrm{a}_2,\cdots,\mathrm{a}_n}^\top$ 相同, 其中 $n$ 为主干网络输出的维度。
分支网络的输出通过内积与主干网络的输出相结合, 以产生三个随机力分量的频率谱, 即 $P_{ii}=\sum_{j=1}^{n}\mathrm{a}_{j}\mathrm{b}_{ij}$。
输入的平均速度剖面和查询点分别使用平均体速度 $U_\text{m}$ 的外特征速度尺度和槽道半高度 $h$ 的外特征长度尺度进行无量纲化。
选择此无量纲方法是因为当用外尺度单位测量时, 不同Reynolds数下的大尺度运动具有相似的特征尺度 \citep{Cossu_2016}。
输入涡黏性被无量纲化为 $\nu_t/\nu Re_\tau^{-1}$, 因为这种缩放方式能使不同Reynolds数下的涡黏性剖面紧密对齐~\citep{Symon_2023}。


\begin{figure}
    \centering
    \includegraphics[width=0.6\linewidth]{Img/chap-4/fig1.png}
    \bicaption{基于DeepONet的神经算子架构, 用于建模随机力谱。
        DeepONet由一个分支网络和一个主干网络组成。
        分支网络接收输入剖面 $U(y)$ 和 $\nu_t(y)$, 而主干网络处理波数-频率-位置四元组 $(k_x h,k_z h,\omega h/U_\text{m},1-\abs{y}/h)$。
        分支网络输出被分成三组, 每组通过内积与主干网络输出结合, 从而产生输出函数 $P_{ii},i=1,2,3$ 的三个分量。}{
        The architecture of the DeepONet-based neural operator for modeling the stochastic forcing spectra.
        The DeepONet consists of a branch network and a trunk network.
        The branch network takes the input profiles $U(y)$ and $\nu_t(y)$, while the trunk network processes the wavenumber-frequency-position tuple $(k_x h,k_z h,\omega h/U_\text{m},1-\abs{y}/h)$.
        The branch output is split into three groups, each combined with the trunk output via inner products to produce three components of the output functions $P_{ii},i=1,2,3$.}
    \label{fig: neural-operator-schem}
\end{figure}

\subsection{神经算子增强的预解预测}

我们提出了神经算子增强的预解模型来预测湍流槽道流动中的速度谱。
该模型通过将线性预解算子与基于神经算子的随机力相结合来实现。
具体而言, 基于DeepONet的神经算子以平均速度和涡黏性作为模型输入, 并输出时域有色随机力在不同波数和壁面法向位置处的频率谱。
进一步地, 获得的强迫谱可通过预解算子传递至速度响应。
如此, 学习到的神经算子与预解算子相结合, 便可将RANS提供的涡黏性和平均速度映射到大尺度流动结构特征尺度下的速度谱。


图~\ref{fig: RNO} 绘制了示意图以说明所提出的神经算子增强预解模型。
具体地, 首先使用耦合了湍流模型的RANS方法来提供涡黏性 $\nu_t$ 及相应的平均速度 $U$。
一方面, 获得的平均速度和涡黏性用于基于方程~\eqref{equ: resolvent-operator} 确定涡黏性强化的预解算子 $\bm{R}$。
另一方面, 它们作为基于DeepONet的强迫模型的输入, 该模型在待评估的位置点产生强迫谱 $\bm{P}$。
最后, 速度谱可通过预解算子 $\bm{R}$ 与神经算子预测的强迫谱 $\bm{P}$ 的内积得到。
所提出的强迫模型能够准确预测脉动速度谱, 且不依赖于RANS模型。
此外, 该方法避免了需要现有速度数据, 例如扫掠增强强迫模型~\citep{Wu_He_2023} 中所需的扫掠速度或涡旋阻尼长度尺度。



\begin{figure}
    \centering
    \includegraphics[width=\linewidth]{Img/chap-4/fig2.png}
    \bicaption{结合基于神经算子的时域有色强迫预测速度谱的示意图。
        RANS模块提供涡黏性 $\nu_t$ 和平均速度 $U$, 用于构建预解算子 $\bm{R}$ 和基于神经算子的强迫模型。
        学习到的神经算子预测随机力谱 $\bm{P}$, 该强迫谱与预解算子 $\bm{R}$ 相结合, 生成速度谱 $\bm{S}$。}{Schematic plot of the velocity spectra predicted with the neural operator-based colored-in-time forcing.
        The RANS module provides the eddy viscosity $\nu_t$  and mean velocity $U$ to construct both the resolvent operator $\bm{R}$ and the neural operator-based forcing.
        The learned neural operator predicts the stochastic forcing spectra $\bm{P}$, which is combined with the resolvent operator~$\bm{R}$ to generate the velocity spectra $\bm{S}$.
    }
    \label{fig: RNO}
\end{figure}


基于DeepONet的强迫模型的神经网络架构描述如下。
主干网络接收一个4维输入, 经过12个隐藏层处理(每层192个神经元, 使用ReLU激活) , 产生一个64维输出。
分支网络具有一个130维输入, 通过相同的隐藏结构(12层$\times$192个神经元, ReLU激活)传递, 并产生一个192维输出。
本研究使用的代码以及学习到的随机力模型已公开可访问~\citep{rno-git}。
需注意, 神经网络的预测可靠性可能受其黑盒性质所限。
尽管如此, 鉴于其非线性表达能力和高训练效率, 神经网络在从高保真数据中提取底层函数映射方面仍具有极大价值。
可解释性可通过可解释性技术进一步获得, 例如特征重要性分析和符号学习方法。
例如, 可以基于Shapley加性解释量化每个输入特征对学习到的函数映射的贡献度, 从而突出最具影响力的特征~\citep{Lundberg2017,WU2023}。
或者, 也可以使用符号回归将黑盒函数映射转化为可解释的白盒模型~\citep{wu}。


在本工作中, 我们的目标是改进随机力的建模, 而非涡黏性。
改进涡黏性也能从两个方面增强基于预解的预测。
一方面, 涡黏性通过RANS方程与平均速度相关联。
改进基于RANS的涡黏性可以在线性算子中提供准确的平均速度。
另一方面, 涡黏性用于部分建模非线性强迫并强化基于预解的线性算子。
人们可以优化涡黏性, 使得主导预解模态与高保真数据的SPOD模态之间达到最佳对齐~\citep{Pickering_Rigas_Schmidt_Sipp_Colonius_2021}。
然而, 即使使用最优涡黏性, 速度响应仍可能表现出较大偏差, 特别是在谱域。
这是因为所提供的涡黏性不是波数或频率的函数, 这与从速度到非线性项的映射本质上是逐尺度相关的事实相矛盾~\citep{Ying_Fu_2024}。
此外, 涡黏性强化的线性模型结合白噪声强迫会产生具有无限频率带宽的速度谱, 导致Taylor时间微尺度消失。
而且, ~\citet{Russo_Luchini_2016} 指出, 没有涡黏模型能够准确复现速度对体积力的响应。
基于这些原因, 本研究主要聚焦于随机力的建模, 而非涡黏性。
非线性强迫和平均速度产生的模型误差在本文中通过随机力进行补偿。
通过这种方式, 时域有色的非线性强迫可以被完全重构, 从而提供准确的速度谱。
此外, 引入神经算子来构建从不同RANS输入到相应随机力的函数映射。
如此, 通过考虑RANS所提供涡黏性的不确定性, 非线性强迫可以相对独立于RANS模型的输入。

\section{数据生成}
\label{sec: data}

\subsection{直接数值模拟}
为训练神经算子模型, 我们进行了直接数值模拟(DNS)以提供随机力谱的训练数据。
计算域在流向、壁面法向和展向的尺寸分别为 $L_x, 2h, L_z$。
壁面平行方向施加周期性边界条件, 上壁面和下壁面则应用无滑移无穿透边界条件。
时间离散采用三阶、多步、刚性稳定方法~\citep{Karniadakis_1991}。
非线性对流项使用显式三阶Adams-Bashforth格式~\citep{Willoughby} 近似, 粘性项则使用隐式三阶Adams-Moulton格式~\citep{Willoughby} 近似。
空间离散方面, 流动变量在流向和展向通过Fourier级数离散, 在壁面法向通过切比雪夫多项式离散。
为减轻非线性项中的混淆误差, 应用了 $3/2$ 规则~\citep{Orszag_1971}。
模拟采用 $N_x\times N_y\times N_z$ 网格点进行。
采用基于消息传递接口(MPI)的二维分解并行策略~\citep{Moser_2013}, 将三维计算域沿两个轴向划分。
为在 $x$-$z$ 平面上使用反混淆的FFT/IFFT计算非线性项, 数据在正变换和逆变换的进程之间进行转置, 以匹配FFT维度。
$Re_\tau=180$ 的模拟在 $128$ 个MPI进程上运行, $Re_\tau=550$ 的模拟在 $384$ 个进程上运行。
所用的计算服务器配备六颗AMD EPYC 7662 CPU (每颗64核, 基频 $\SI{2.0}{GHz}$, 禁用超线程) , 提供384个进程。

本工作使用了两个槽道流动案例的DNS数据, 对应的摩擦Reynolds数分别为 $Re_\tau=180$ 和 $Re_\tau=550$。
此处, $Re_\tau=u_\tau h/\nu$ 基于摩擦速度 $u_\tau$ 和槽道半高度 $h$ 定义。
$Re_\tau=180$ 的流动数据用于训练神经算子模型, $Re_\tau=550$ 的数据用于测试学习模型的泛化能力。
案例设置的详细信息列于表~\ref{tab:dns_records}。
本DNS与参考数据~\citep{Lee_Moser_2015} 的平均速度和Reynolds正应力对比如图~\ref{fig: V-S1S2-Retau-180} 所示。
平均速度和Reynolds应力以粘性单位绘制, 其结果与 \citet{Lee_Moser_2015} 的结果吻合良好。
该DNS求解器及数据处理流程已在先前研究中得到验证~\citep{Wu_He_2020, Wu_2021_PRF, Wu_He_2023}。


时间Fourier变换使用加窗FFT计算。
In the DNS, flow snapshots are recorded after the flow reaches a statistically steady state.
The time Fourier transformation is performed using Welch's method with a Hann window and $75\%$ overlap.
A $\sqrt{8/3}$ correction factor is applied to the frequency modes to ensure energy conservation.
The Hann window spans $N$ snapshots, providing a frequency resolution of $\Delta\omega=2\pi/(N\Delta t_\text{sample})$, with the frequency range extending from~$-\tfrac{N}{2}\Delta\omega$ to~$(\tfrac{N}{2}-1)\Delta\omega$.
The frequency range must be adequately wide to fully encompass the spectral energy of the flow variables, such that their spectra diminish to negligible levels at the range boundaries for all relevant wavenumbers and wall-normal positions.
For both the $Re_\tau=180$ and $Re_\tau=550$ cases, flow snapshots are sampled at uniform time intervals of $\Delta t_\text{sample}=\num{5.e-2}h/U_\text{m}$.
A Hann window of length $N=\num{1024}$ snapshots is applied, corresponding to a time duration of $T=51.2 h/U_\text{m}$.
A total of $\num{10001}$ snapshots and $\num{5400}$ snapshots are collected for the $Re_\tau=180$ case and the $Re_\tau=550$ case, respectively.


\begin{table}
    \begin{center}
        \bicaption{用于生成高保真数据的DNS案例设置。
            $Re_\tau=180$ 案例用于训练神经算子模型, 而 $Re_\tau=550$ 案例用于泛化能力测试。
            $L_x$ 和 $L_z$ 分别表示流向和展向的计算域尺寸。
            $\Delta_x^+$ 和 $\Delta_z^+$ 为壁面平行方向对应的网格间距。
            $\Delta y_w^+$ 和 $\Delta y_c^+$ 分别为壁面处和槽道中心处的壁面法向网格间距。
            $\Delta t$ 表示时间步长。
            带 $^+$ 上标的尺度以壁面单位度量。}{Case setup of the DNS for generating high-fidelity data.
            The case of $Re_\tau=180$ is used for training the neural operator model, while the case of $Re_\tau=550$ is for the generalizability test.
            $L_x$ and $L_z$ represent the computational domain sizes in streamwise and spanwise directions, respectively.
            $\Delta_x^+$ and $\Delta_z^+$ are the corresponding grid spacings in the wall-parallel direction.
            $\Delta y_w^+$ and $\Delta y_c^+$ are the wall-normal grid spacing at the wall and channel center.
            $\Delta t$ denotes the time step.
            The scale with notation $^+$ is measured in wall units.}
        \begin{tabular}{ccccccccccc}
            \toprule
            $Re_\tau$ & $L_x/h$ & $L_z/h$ & $N_x$ & $N_y$ & $N_z$ & $\Delta x^+$ & $\Delta z^+$ & $\Delta y_w^+$ & $\Delta y_c^+$ & $\Delta{t}U_\text{m}/h$ \\
            \midrule
            180       & $8\pi$  & $4\pi$  & 384   & 129   & 384   & 11.8         & 5.89         & 0.0542         & 4.42           & 0.005                   \\
            550       & $4\pi$  & $2\pi$  & 576   & 257   & 576   & 12.0         & 6.00         & 0.0414         & 6.75           & 0.001                   \\
            \bottomrule
        \end{tabular}
        \label{tab:dns_records}
    \end{center}
\end{table}

\begin{figure}
    \centering
    \includegraphics[width=\linewidth]{Img/chap-4/fig3.eps}
    \bicaption{本研究DNS结果(实线)与来自~\citet{Lee_Moser_2015} 的参考数据(标记符号)在 $Re_\tau=180$(灰色)和 $550$(黑色)下的对比。(a) 平均速度 $U^+$; (b) 雷诺正应力:$\abra{uu}^+$、$\abra{vv}^+$ 和 $\abra{ww}^+$。}{Comparison of the present DNS results (solid lines) with reference data (marker symbols) from~\citet{Lee_Moser_2015}  at $Re_\tau=180$ (gray) and $550$ (black). (a) Mean velocities $U^+$; (b) Reynolds normal stresses: $\abra{uu}^+$, $\abra{vv}^+$, and $\abra{ww}^+$.}
    \label{fig: V-S1S2-Retau-180}
\end{figure}

\subsection{槽道湍流特征尺度}
湍流流动表现出显著的尺度分离现象~\citep{Smits_2011}, 这对模拟所有流动尺度构成了重大挑战。
特征尺度下的相干结构已被认为是自维持过程的核心要素~\citep{Cossu_2016}。
因此, 在本工作中, 我们旨在利用所提出的神经算子模型来预测与自维持含能运动相关联的特征尺度下的速度谱。

已有研究识别出多种特征尺度下的流动结构, 以揭示大尺度运动的自相似本质~\citep{Hwang_PhysRevLett_2010,Hwang_2015,Cossu_2016}。
例如, 长条带结构和短而高的涡旋结构(通常称为发夹涡包)被识别为两种动态互连的含能运动。
这些运动总是共同出现, 表明它们是单个附着涡涡的动态关联组成部分。
长条带主要携带流向湍动能, 并表现出自相似标度律, 其尺寸和几何形状可近似表示为:
\begin{equation}
    y\simeq 0.1\lambda_z,\quad \lambda_x\simeq 10\lambda_z \text{,}
\end{equation}
其中 $\lambda_x=2\pi/k_x$ 和 $\lambda_z=2\pi/k_z$ 分别表示流向波长和展向波长。
短而高的涡包则对各速度分量的湍动能均有贡献, 其特征可描述为:
\begin{equation}
    y\simeq 0.5\sim 0.7\lambda_z,\quad \lambda_x\simeq 2\sim 3 \lambda_z \text{.}
\end{equation}
展向长度尺度 $\lambda_z \simeq 1.5h$ 对应于在过度阻尼的大涡模拟(其中小尺度结构被抑制)中流向速度的展向预乘谱 $k_z\abra{\hat{u}\hat{u}^\herm}$ 的峰值位置~\citep{Hwang_PhysRevLett_2010}。
在此展向尺度下的长条带和涡包分别被称为超大尺度运动(VLSMs)和大尺度运动(LSMs) ~\citep{Hwang_PhysRevLett_2010, Hwang_2015}, 其表现出以下尺度特征:
\begin{subequations}
    \label{equ: similar scaling}
    \begin{gather}
        y\simeq 0.15h,\quad \lambda_z\simeq 1.5h,\quad \lambda_x\simeq 15h \quad \text{(VLSMs)},\\
        y\simeq 0.75\sim 1h,\quad \lambda_z\simeq 1.5h,\quad \lambda_x\simeq 3\sim 4 h \quad \text{(LSMs)}.
    \end{gather}
\end{subequations}

本工作中研究的相干结构的特征尺度如图~\ref{fig: characteristic-scales} 所示。
展向尺度 $\lambda_z\simeq 1.5h$ 对应的展向波数约为 $k_zh\simeq 4.2$。
在此, 我们聚焦于约 $k_zh\simeq 4\sim 6$ 的展向波数, 如图~\ref{fig: characteristic-scales} 中阴影区域所示, 该范围由DNS中与式~\eqref{equ: similar scaling} 相似尺度最匹配的离散波数确定。
短而高的涡包对应于 $\lambda_x/\lambda_z=2,2.5,3$, 而长条带结构则遵循 $\lambda_x/\lambda_z=10$。
这些特征尺度在图~\ref{fig: characteristic-scales} 中以圆圈标出。
对于训练数据, 我们在 $Re_\tau=180$ 处选择以下尺度: 长条带结构为 $\bm{k}h=(0.5,5)$; 短而高的涡包结构为 $\bm{k}h=(1.5,4.5),(2,6),(2.25,4.5),(2.5,5),(2.75,5.5),(3,6)$。
这些尺度在图~\ref{fig: characteristic-scales}(a)中标记为红色圆圈。
其余的尺度(在图~\ref{fig: characteristic-scales} 中以蓝色圆圈标记, 包括 $Re_\tau=180$ 和 $550$ 的情况)将用于学习模型的泛化测试。

\begin{figure}
    \centering
    \includegraphics[width=\linewidth]{Img/chap-4/fig4.eps}
    \bicaption{湍流槽道流动中长条带结构及短而高的涡包结构在 (a) $Re_\tau=180$ 和 (b) $Re_\tau=550$ 下的特征尺度。
        实线表示展弦比 $\lambda_x/\lambda_z=2,2.5,3,10$, 灰色网格表示波数分辨率。
        圆圈表示特征尺度, 其中短而高的涡包结构对应 $\lambda_x/\lambda_z=2,2.5,3$, 长条带结构对应 $\lambda_x/\lambda_z=10$。
        灰色阴影区域表示本工作中研究的展向波数范围。
        红色圆圈代表用于模型训练的尺度, 蓝色圆圈表示用于泛化测试的尺度。}{Characteristic scales of the long streaks and the short and tall vortex packets in turbulent channel flows at (a) $Re_\tau=180$ and (b) $Re_\tau=550$.
        Solid lines show aspect ratios $\lambda_x/\lambda_z=2,2.5,3,10$, and the gray mesh indicates wavenumber resolution.
        Circles denote the characteristic scales, where the short and tall vortex packets correspond to $\lambda_x/\lambda_z=2,2.5,3$ and long streaky structures correspond to $\lambda_x/\lambda_z=10$.
        The gray shaded zone indicates the range of spanwise wavenumber investigated in this work.
        The red circles represent the scales for model training, and the blue circles indicate the scales for generalization tests.}
    \label{fig: characteristic-scales}
\end{figure}

\subsection{基于神经算子的强迫模型的训练数据}\label{sec: training data}
在此, 我们概述了为构建基于神经算子的强迫模型而收集训练数据的流程。
每个数据点的结构为三元组, 即:
\begin{equation}
    \Bigg[
    \underbrace{\frac{U(y)}{U_\text{m}},\frac{1}{Re_\tau}\frac{\nu_t(y)}{\nu},}_{\text{branch net input}}\quad
    \underbrace{k_xh,k_zh,\frac{\omega h}{U_\text{m}},1-\frac{\abs{y}}{h},}_{\text{trunk net input}}\quad
    \underbrace{P_{11},P_{22},P_{33}}_{\text{output}}\,
    \Bigg].
\end{equation}
输入函数 $U(y)$ 和 $\nu_t(y)$ 是相互对应的。
为了构建无限函数空间之间的映射, 需要对任意涡黏性剖面进行采样以表征该函数空间。
我们遵循~\citet{lu2021learning} 的工作, 采用正交多项式来实现这一目标。
具体而言, 在Cess涡黏性剖面(见附录~\ref{app: RANS} 描述)上添加一个扰动 $\delta(y)$, 即:
\begin{equation}
    \nu_t(y)=\nu_t^{\text{Cess}}+\delta(y),
\end{equation}
其中 $\delta(y)$ 使用修正的正交Chebyshev多项式生成, 如~\citep{lu2021learning} 所示:
\begin{equation}
    \frac{1}{Re_\tau}\frac{\delta(y)}{\nu}=\sum_{n=0}^N a_n \cos\bra{\frac{\pi}{2}\frac{y}{h}} \cos\bra{n\arccos \frac{y}{h}} \text{.}
\end{equation}
调制函数 $\cos\bra{\frac{\pi}{2}\frac{y}{h}}$ 的作用是减小近壁处的扰动幅值, 避免出现非物理值(例如负涡黏性) , 并使扰动在边界处消失, 从而满足 $\nu_t(\pm h)=0$。
通过将奇数 $n$ 对应的系数 $a_n$ 设为零以满足对称性约束, 最高多项式阶数为 $N=16$。
在本工作中, 系数从区间 $\sbra{-\bar{a},\bar{a}}$ 上的均匀分布~\citep{YIN2022} 中随机采样, 其中选取 $\bar{a}=0.012$ 以确保DNS和RANS模型的涡黏性均位于采样涡黏性的覆盖范围内。

本工作中, 神经网络在包含 64 个 $Re_\tau=180$ 下采样涡黏性剖面的数据集上进行训练。
图~\ref{fig: random-nut-U} 将用于模型训练的采样涡黏性剖面与Cess模型和DNS的涡黏性剖面进行了比较, 同时展示了通过求解RANS动量方程得到的相应平均速度。
DNS导出的涡黏性通过将Reynolds剪切应力 $\abra{uv}^\text{(DNS)}$ 投影到平均流向速度的壁面法向梯度 $\partial_y U^{\text{(DNS)}}$ 上得到, 即 $\abra{uv}^\text{(DNS)}=-\nu_t^{\text{(DNS)}} \partial_y U^{\text{(DNS)}}$。
可以看出, 生成的采样涡黏性剖面和平均速度涵盖了一个范围, 该范围包含了DNS和Cess模型的结果。

\begin{figure}
    \centering
    \includegraphics[width=\linewidth]{Img/chap-4/fig5.eps}
    \bicaption{(a) 由正交多项式生成的采样涡黏性剖面(灰色线)及 (b) 相应的平均速度剖面。图中也展示了DNS($Re_{\tau}=180$ 为黑色实线, $Re_{\tau}=550$ 为黑色虚线)和Cess模型(红色虚线)的涡黏性和速度剖面以作对比。}{(a) The sampled eddy-viscosity profiles (gray lines) generated by orthogonal polynomials and (b) the corresponding mean velocity profiles. The eddy viscosity and velocity profiles from the DNS (black solid line for $Re_{\tau}=180$ and black dashed line for $Re_{\tau}=550$) and Cess model (red dashed line) are also presented for comparison.}
    \label{fig: random-nut-U}
\end{figure}

基于采样的涡黏性剖面, 利用第~\ref{sec: Modeling of stochastic forcing spectra} 节中描述的DNS速度交叉谱, 得到目标随机力谱。
目标随机力谱在不同尺度下获得, 对应频率范围 $\omega h/U_{\text{m}}\in [-20\pi, 20\pi]$ 内的每个离散频率。
我们采用的频率分辨率为 $\Delta\omega h/U_\text{m}=\pi/51.2$, 共产生 2048 个离散频率。
为在保证充分采样的同时控制训练数据集规模的可管理性, 我们采用Ramer-Douglas-Peucker算法~\citep{Douglas_2011} 进行动态频率采样。
该方法在谱峰附近(能量集中区域)提供精细采样, 在高频区域(能量衰减区域)则采样较稀疏。
壁面法向位置的采样采用与DNS网格点相同的分辨率。
基于对称性考虑, 仅假设槽道高度的一半(即从壁面到槽道中心)用于分析。

基于生成的数据, 第~\ref{sec: method} 节描述的基于DeepONet的模型通过最小化强迫预测的均方误差(MSE)进行训练。
训练使用Adam优化器, 初始学习率为 $\num{1.e-3}$。
学习率根据验证损失动态调整: 若验证损失连续 3 个epoch停滞, 则学习率减半, 以改善训练收敛性。
训练过程共进行 2000 个epoch, 批量大小为 2048。
所有权重均通过Xavier初始化, 以实现更快的训练收敛~\citep{Xavier_2010}。

\section{结果}\label{sec: results}
在本节中, 我们展示神经算子在槽道流动相干结构特征尺度下对随机力和速度的预测能力。
我们首先通过将神经算子与DNS目标数据进行比较, 展示其在随机力方面的训练性能。
接着, 我们展示使用学习到的神经算子以及白噪声随机力进行速度谱预测的结果。
随后, 展示学习模型在不同波数下结合多种涡黏模型(包括Cess模型、Spalart-Allmaras模型和 $k$-$\omega$ 模型)的预测性能。
这些RANS模型在湍流槽道流动中的细节见附录~\ref{app: RANS}。
最后, 在摩擦Reynolds数 $Re_\tau=550$ 的湍流槽道流动中对学习模型进行泛化测试。

\subsection{神经算子随机力预测结果}
图~\ref{fig:fx_peak} 展示了在 $Re_\tau=180$ 下, 随机力谱在频率和壁面法向位置上的谱峰值 $\max(\bm{P})$, 包括目标值(DNS)和基于神经算子的预测值。
图中给出了训练波数和测试波数的结果。
学习模型的预测是使用Cess模型提供的涡黏性和平均速度剖面得到的。
学习模型准确地捕捉了所有特征尺度下的谱峰值。
仅在测试尺度 $\bm{k}h=(2.25,5.5)$ (对应于 $\abs{\bm{k}}h\approx 5.9$)处存在微小差异。
\begin{figure}
    \centering
    \includegraphics[width=\linewidth]{Img/chap-4/fig6.eps}
    \bicaption{目标随机力谱与学习模型预测在 $Re_\tau=180$ 下的谱最大值。
        实心与空心标记分别表示训练波数和测试波数。}{The maximum values of the stochastic forcing spectra from the target and the learned predictions at $Re_\tau=180$.
        Filled and hollow markers denote training and testing wavenumbers, respectively.}
    \label{fig:fx_peak}
\end{figure}

图~\ref{fig: kx-08-kz-08-Cess-Retau-180-stf-psd} 展示了在 $Re_\tau=180$ 下, 波数 $\bm{k}h=(2,4)$ 处随机力的频率谱, 包括目标值(DNS)和学习模型的预测值。
此展向波数大致对应于大尺度结构运动。
强迫谱 $P_{ii}$ 通过谱峰值进行归一化, 使得能量最高的强迫分量峰值为1, 以展示强迫谱的相对分布。
需注意, 白噪声随机力的谱在所有频率和壁面法向位置上为常数, 因此为简洁起见未予展示。
可以看出, 流向强迫分量的强度主导于壁面法向和展向分量。
随机力谱在壁面法向位置约 $y^+\simeq 40$ (流向) 、$y^+\simeq 80$ (壁面法向)和 $y^+\simeq 50$ (展向)处达到峰值。
强迫谱密度在远离峰值频率和峰值壁面法向位置时单调衰减。
基于神经算子的模型准确地预测了频率和壁面法向位置的峰值位置。
此外, 图~\ref{fig: kx-08-kz-08-Cess-Retau-180-stf-psd} 中的等值线表示在对数尺度上均匀分布在 $\num{1.e-3}$ 到 $1$ 之间的不同能级, 这表明学习模型能够准确捕捉频率和壁面法向位置的衰减行为。

\begin{figure}
    \centering
    \includegraphics[width=\linewidth]{Img/chap-4/fig7.png}
    \bicaption{在波数 $\bm{k}h=(2,4)$、$Re_\tau=180$ 条件下, 随机力频率谱 $P_{ii},(i=1,2,3)$ 的对比:(a--c) 目标谱与 (d--f) 使用Cess模型的神经算子预测谱。第一列 (a,d)、第二列 (b,e) 和第三列 (c,f) 分别对应流向、壁面法向和展向分量。
        实线等值线表示目标谱, 虚线等值线表示学习模型的预测谱。}{Comparison of the frequency spectra $P_{ii},(i=1,2,3)$ of stochastic forcing at the wavenumber $\bm{k}h=(2,4)$ for $Re_\tau=180$ between (a--c) the target and (d--f) the neural operator predictions using the Cess model. The first (a,d), second (b,e), and third (c,f) columns correspond to the streamwise, wall-normal, and spanwise components, respectively.
        Solid isolines denote the target spectra, and dashed isolines denote the predictions of the learned model.}
    \label{fig: kx-08-kz-08-Cess-Retau-180-stf-psd}
\end{figure}

神经算子的预测能力也在不同波数下的随机力谱方面进行了检验。
图~\ref{fig: Retau-180-stf-psd} 展示了Reynolds数 $Re_\tau=180$ 的流动在展向波数 $k_zh=4.5$ 和尺度比 $\lambda_x/\lambda_z = 9,3,2.5,2$ 下的强迫谱, 并与基于方程~\eqref{equ: P} 的DNS目标谱进行了比较。
对比在两个壁面法向位置进行, 分别对应谱强度最大的位置和槽道中心位置。
值得注意的是, 各强迫分量的峰值频率在不同波数下均接近 $k_x U_\text{m}$, 这表明对流速度(即峰值频率与相应流向波数的比值~\citep{Jimenez_2009})在本案例中与 $U_\text{m}$ 一致。
流向随机力分量在强度上比其他分量高出几个数量级。
壁面平行强迫分量在槽道中心的谱强度相比其峰值显著降低, 而壁面法向谱强度在槽道中心仍接近其峰值强度。
在两个壁面法向位置, 学习到的神经算子模型预测的强迫谱均与不同波数下的目标数据高度吻合。

\begin{figure}
    \centering
    \includegraphics[width=\linewidth]{Img/chap-4/fig8.eps}
    \bicaption{$Re_\tau=180$ 条件下随机力频率谱的对比, 展示了目标数据与使用Cess模型的学习模型预测的强迫谱 $P_{ii}$。
        谱已通过其最大值 $\max{(\bm{P})}$ 归一化。
        所有谱图均在相同展向波数 $k_zh=4.5$ 下绘制, 流向波数对应尺度比为 (a) $\lambda_x/\lambda_z=9$、(b) $3$、(c) $2.5$ 和 (d) $2$。
        每列分别展示流向、壁面法向和展向强迫分量。
        频率谱在峰值壁面法向位置和槽道中心两处进行评估。
        黑色虚线表示频率 $k_xU_\text{m}$。}{
        Frequency spectra of the stochastic forcing for $Re_\tau=180$, comparing the forcing spectra $P_{ii}$ of the target data and the learned model prediction using the Cess model.
        The spectra are normalized by their maximum values $\max{(\bm{P})}$.
        All plots are at the same spanwise wavenumber $k_zh=4.5$, with streamwise wavenumbers corresponding to scale ratios of (a) $\lambda_x/\lambda_z=9$, (b) $3$, (c) $2.5$, and (d) $2$.
        Each column presents streamwise, wall-normal, and spanwise forcing components, respectively.
        The frequency spectra are evaluated at both the peak wall-normal location and the channel center.
        The black dashed line indicates the frequency $k_xU_\text{m}$.
    }
    \label{fig: Retau-180-stf-psd}
\end{figure}

为量化学习到的神经算子模型的预测精度, 我们计算了目标随机力谱与预测谱之间的相对误差:
\begin{equation}\label{equ: err_stf}
    \mathcal{E}_{P_{ii}}=
    \int_{-\infty}^{\infty}\iint_{-h}^{h} \abs{
        \frac{P_{ii}}{\sbra{\bm{P}}}-
        \frac{P_{ii}^\text{(TGT)}}{\sbra{\bm{P}^\text{(TGT)}}}
    }\delta(y-y') \dif{y}\dif{y'}\dif{\omega} \text{,}
\end{equation}
其中 $P_{ii}^{\text{(TGT)}}$ 表示基于方程~\eqref{equ: P} 计算的目标谱, $P_{ii}$ 代表神经算子的预测值, 并且
\begin{equation}
    \sbra{\bm{P}} = \int_{-\infty}^{\infty} \|\bm{P} \|_*\dif{\omega}
\end{equation}
是一个能量范数~\citep{Jovanovic_Bamieh_2005}, 其定义为在指定波数下, 频率和壁面法向位置上的积分谱能量。
根据方程~\eqref{equ: RPR} 中的输入-输出关系, 速度对随机力强度呈线性响应。
因此, 方程~\eqref{equ: err_stf} 中定义的误差度量用于量化各强迫分量的相对误差, 该度量也被~\citet{Holford_Lee_Hwang_2023} 采用。

图~\ref{fig: stf-error-180-Cess} 展示了使用Cess涡黏性剖面时, 神经算子模型在强迫谱上的预测误差, 其中训练尺度用实心三角形标记。
可以看出, 流向分量的预测误差在 $8\%$ 以内, 壁面法向分量在 $2\%$ 以内, 展向分量在 $4\%$ 以内。
此外, 在未见波数上的预测误差与训练波数上的误差相当, 这表明学习到的强迫模型可以应用于训练尺度附近的不同波数。
同时, Cess涡黏性及相应的平均速度并未出现在训练数据集中, 这表明学习到的神经算子可以应用于未见过的RANS模型。
关于模型在不同波数和涡黏模型之间对速度统计量的泛化能力, 将在第~\ref{sec: generalization} 节中讨论。

\begin{figure}
    \centering
    \includegraphics[width=\linewidth]{Img/chap-4/fig9.eps}
    \bicaption{基于神经算子的预测与目标随机力谱之间的相对误差 $\mathcal{E}_{P_{ii}}$, 针对 (a) $P_{11}$、(b) $P_{22}$ 和 (c) $P_{33}$, 在 $Re_\tau=180$ 下使用Cess模型于不同特征尺度处计算。
    误差基于式~\eqref{equ: err_stf} 计算。
    实心与空心标记分别表示训练尺度和测试尺度。}{Relative error $\mathcal{E}_{P_{ii}}$ between the neural operator-based predictions and the target stochastic forcing spectra in terms of (a) $P_{11}$, (b) $P_{22}$, and (c) $P_{33}$ with $Re_\tau=180$ using Cess model at different characteristic scales.
    The error is computed based on Eq.~\eqref{equ: err_stf}.
    The filled and hollow markers denote the training and testing scales, respectively.}
    \label{fig: stf-error-180-Cess}
\end{figure}

\subsection{速度谱预测}
所提出的基于神经算子的强迫模型能够在速度谱上提供准确的预测。
图~\ref{fig: kx-10-kz-12-Retau-180-Cess-vel} 清楚地展示了这一点, 其中将学习模型预测的速度谱与DNS结果以及白噪声预测在Reynolds数 $Re_\tau=180$ 的槽道流动中波数 $\bm{k}h=(2,4)$ 处进行了对比。
需注意, Cess涡黏模型是针对高Reynolds数流动调优的~\citep{Cess_1958}, 其在低至中等Reynolds数(例如 $Re_\tau=180$, $550$)下的精度并无保证~\citep{Morra_Henningson_2021}。
鉴于此, 我们采用结合Cess涡黏性剖面的学习强迫模型, 同时为白噪声强迫模型配备DNS导出的涡黏性剖面(后者被认为是模拟槽道流动Reynolds应力最精确的涡黏性) 。
这突显了所提出的基于神经算子的强迫的预测性能, 其独立于所采用的涡黏模型。
在给定建模强迫谱 $\bm{P}$ 的情况下, 使用方程~\eqref{equ: RPR} 计算速度谱。
当前速度谱通过 $\sbra{\bm{S}}$ 归一化, 以描绘在指定波数下频率 $\omega$ 和壁面法向位置 $y$ 上的谱能量密度分布。
对于固定的壁面法向位置, 谱能量集中在以峰值频率 $k_xU_c$ 为中心的狭窄频带内。
在槽道流动中, 对流速度 $U_c$ 在缓冲层及以上区域接近平均速度 $U$ (如图~\ref{fig: kx-10-kz-12-Retau-180-Cess-vel} 中红色虚线所示) , 而在粘性子层内保持恒定~\citep{Jimenez_2009}。
壁面平行速度分量的谱峰值位于 $y^+\simeq 30\sim 40$, 壁面法向速度的谱峰值位于 $y^+\simeq 70$ 附近。
可以看出, 学习到的强迫模型正确捕捉了峰值频率和峰值壁面法向位置。
相比之下, 白噪声随机力明显低估了壁面平行速度分量的峰值频率。
此外, 白噪声低估了壁面平行速度谱在 $y^+\simeq 10\sim 20$ 处的峰值位置, 并高估了壁面法向速度谱在 $y^+\simeq 180$ 处(即槽道中心)的峰值位置。
图~\ref{fig: kx-10-kz-12-Retau-180-Cess-vel} 中的等值线对应于以对数间隔 $\num{1.e-1}$ 均匀分布的等高线。
学习预测与DNS结果之间的高度一致性表明, 基于神经算子的强迫模型准确预测了各速度分量的频率带宽和壁面法向能量分布。
相比之下, 白噪声强迫模型高估了频率带宽, 导致时间动力学预测不准确。
根据我们的数值实验, 学习模型和白噪声强迫对非对角元素 $\langle\tilde{u}\tilde{v}^\text{H}\rangle$ 的预测结果相似, 因此此处省略其图示。

\begin{figure}
    \centering
    \includegraphics[width=\linewidth]{Img/chap-4/fig10.png}
    \bicaption{波数 $\bm{k}h=(2,4)$、$Re_\tau=180$ 条件下的归一化速度频率谱:(a--c) $\tilde{u}$、(d--f) $\tilde{v}$ 和 (g--i) $\tilde{w}$。
        结果分别展示 (a,d,g) DNS、(b,e,h) 学习模型 和 (c,f,i) 白噪声(w.n.)预测。
        谱经过归一化, 使其在频率和壁面法向方向上的积分等于一。
        黑色实线和虚线分别表示DNS和预解预测的等值线, 红色虚线表示频率 $k_x U(y)$。
        红色十字标记表示速度谱的峰值位置。}{Normalized frequency spectra of velocity (a--c) $\tilde{u}$ , (d--f) $\tilde{v}$, and (g--i) $\tilde{w}$ at the wavenumber~$\bm{k}h=(2,4)$ with $Re_\tau=180$.
        The results are shown for (a,d,g) DNS, (b,e,h) the learned, and (c,f,i) white-noise (w.n.) predictions.
        The spectra are normalized such that their integral over frequency and wall-normal direction equals to one.
        Solid and dashed black lines indicate isolines of DNS and resolvent predictions, respectively, and the red dashed line denotes the frequency $k_x U(y)$.
        The red crosses indicate the peak location of the velocity spectra.}
    \label{fig: kx-10-kz-12-Retau-180-Cess-vel}
\end{figure}

学习到的强迫模型能够准确预测不同波数下的速度频率谱。
图~\ref{fig: Retau-180-vel-psd} 为此提供了支持, 该图展示了在Reynolds数 $Re_\tau=180$ 的流动中, 学习模型在展向波数 $k_zh=4.5$ 和尺度比 $\lambda_x/\lambda_z=9,3,2.5,2$ 下预测的归一化谱, 并与DNS结果进行了比较。
频率谱绘制在图~\ref{fig: kx-10-kz-12-Retau-180-Cess-vel} 所示的峰值壁面法向位置处。
图~\ref{fig: Retau-180-vel-psd}(a)展示了长条带结构的流向速度谱, 其在对数律区 $y^+\simeq 84$ 处出现谱峰。
图~\ref{fig: Retau-180-vel-psd}(b--d)展示了短而高的涡包结构的谱, 其谱峰分别位于缓冲层和对数律区的 $y^+\simeq 28,62$ 和 $73$。
频率 $k_x U$ 在图~\ref{fig: Retau-180-vel-psd} 中以黑色虚线标出。
可以观察到, 所有三个速度分量的峰值频率均与 $k_xU$ 高度吻合, 证实了对流速度近似于当地平均速度。
流向速度在三个速度分量中, 在不同尺度下均表现出最强的谱强度, 这一点两种强迫模型均能预测。
另一方面, 速度谱在远离峰值频率时迅速衰减。
学习到的强迫模型准确地捕捉了这种衰减行为, 而白噪声随机力在远离峰值频率处提供的谱能量比DNS结果高出几个数量级。

\begin{figure}
    \centering
    \includegraphics[width=\linewidth]{Img/chap-4/fig11.eps}
    \bicaption{$Re_\tau=180$ 流动中速度分量的归一化频率谱 $S_{11}$、$S_{22}$ 和 $S_{33}$ 的对比, 涉及DNS结果以及基于预解方法结合学习强迫模型和白噪声(w.n.)强迫模型的预测。
        频率谱展示在峰值壁面法向位置处。
        波数设定为 $k_zh=4.5$, 箭头表示尺度比 $\lambda_x/\lambda_z=2, 2.5, 3, 9$ 的增大方向。
        各图中的黑色虚线表示对应 $k_xU$ 的频率。}{Normalized frequency spectra of the velocity components for flows of $Re_\tau=180$ in terms of $S_{11}$, $S_{22}$, and $S_{33}$ with comparison between the DNS and the resolvent-based predictions with the learned and white-noise (w.n.) forcing models.
        The frequency spectra are shown at the peak wall-normal position.
        The wavenumber is set as $k_zh=4.5$, and the arrow indicates the increasing scale ratios $\lambda_x/\lambda_z=2, 2.5, 3, 9$.
        The black dashed line in each plot indicates the frequency corresponding to $k_xU$.}
    \label{fig: Retau-180-vel-psd}
\end{figure}

\subsection{时间尺度预测}\label{sec: time scale prediction}
我们检验了学习到的基于神经算子的强迫模型在捕捉大尺度湍流结构时间特征方面的性能。
速度分量的时间相关性可由频率谱得到:
\begin{equation}\label{equ: time-corr}
    R(\bm{k},\tau,y)=\int_{-\infty}^{\infty}\Phi e^{\ii\omega\tau}\dif{\omega},\quad \Phi=\frac{\abra{\tilde{u}\tilde{u}^\herm}}{\int_{-\infty}^{\infty}\abra{\tilde{u}\tilde{u}^\herm}\dif{\omega}} \text{,}
\end{equation}
其中 $\tau$ 表示时间间隔, $\Phi$ 为条件频率谱。
该条件谱是经过给定波数下总能量归一化的波数-频率谱。
这种归一化确保谱在频率上积分为单位1, 从而能够对频率分布进行定量比较~\citep{Wilczek_2023,Marusic_2012}。
流向速度的时间相关性使用式~\eqref{equ: time-corr} 计算, 其他速度分量采用相同的计算方法。
图~\ref{fig: Time-correlation-Retau-180-u} 比较了在 $Re_\tau=180$ 下, 波数 $\bm{k}h=(0.5,5)$ (长条带结构)和 $\bm{k}h=(2,4)$ (短而高的涡包结构)处流向速度时间相关性的幅值 $\abs{R}$, 结果包括DNS、学习模型预测以及白噪声预测。
时间相关性在三个壁面法向位置展示, 即 $y^+=11$ (缓冲层) 、$y^+=80$ (对数律区)和 $y^+=180$ (槽道中心) 。
图~\ref{fig: Time-correlation-Retau-180-u}(a--c)展示了长条带结构的时间相关性。
DNS结果表明, 与缓冲层和对数律区相比, 该相干结构的时间相关性在槽道中心衰减最快。
图~\ref{fig: Time-correlation-Retau-180-u}(d--f)展示了高而短的涡包结构的时间相关性。
可以看出, 缓冲层的时间相关性衰减速率最快, 而对数层和槽道中心附近的时间相关性则表现出相对缓慢的衰减。
学习到的强迫模型准确地预测了时间相关性, 与当前三个壁面法向位置的DNS结果高度一致。
相比之下, 基于白噪声的预测在 $\tau=0$ 处出现急剧下降, 且未能捕捉到三个壁面法向位置的正确衰减行为。
总体而言, 与白噪声强迫相比, 学习到的时域有色强迫模型能够显著改善时间相关性的预测。

\begin{figure}
    \centering
    \includegraphics[width=\linewidth]{Img/chap-4/fig12.eps}
    \bicaption{$\tilde{u}$ 在波数 (a--c) $\bm{k}h=(0.5,5)$ 和 (d--f) $\bm{k}h=(2,4)$、$Re_\tau=180$ 下的时间相关性。
        结果展示在壁面法向位置 (a,d) 缓冲层 $y^+=11$、(b,e) 对数律区 $y^+=80$ 和 (c,f) 槽道中心 $y^+=180$ 处。
        各子图中对比了DNS、学习模型和白噪声(w.n.)模型的结果。}{Time correlation of $\tilde{u}$ at wavenumbers (a--c) $\bm{k}h=(0.5,5)$ and (d--f) $\bm{k}h=(2,4)$ for $Re_\tau=180$.
        The results are shown at wall-normal positions (a,d) $y^+=11$ in the buffer layer, (b,e) $y^+=80$ in the log-law region, and (c,f) $y^+=180$ at the channel center.
        The results of the DNS, the learned, and the white-noise (w.n.) models are compared in each panel.}
    \label{fig: Time-correlation-Retau-180-u}
\end{figure}

为了进一步检验学习强迫模型的时间预测性能, 我们将预测的Taylor时间微尺度与DNS数据进行比较。
Taylor时间微尺度可用于表征时间间隔 $\tau=0$ 附近时间相关性的衰减速率, 其定义为:
\begin{equation}
    \lambda_T=\bra{-\frac{1}{2}\frac{\p^2\abs{R}}{\p \tau^2}\bigg|_{\tau=0}}^{-1/2}.
\end{equation}
其可使用频率谱的一阶和二阶条件矩~\citep{Wu_2021_PRF} 计算得到:
\begin{equation}
    \lambda_T(\bm{k},y)=\sbra{
        \frac{1}{2}\int_{-\infty}^{\infty}\omega^2\Phi\dif{\omega}-\frac{1}{2}\bra{\int_{-\infty}^{\infty}\omega\Phi\dif{\omega}}^2
    }^{-1/2} \text{,}
\end{equation}
其中条件频率谱 $\Phi$ 在式~\eqref{equ: time-corr} 中定义。
图~\ref{fig: Retau-180-Taylor-two-wave} 展示了 $Re_\tau=180$ 下波数 $\bm{k}h=(0.5,5)$ 和 $\bm{k}h=(2,4)$ 处速度分量的Taylor时间微尺度。
$\bm{k}h=(0.5,5)$ 处的长条带结构展现出相对较大的Taylor时间微尺度, 尤其是流向速度分量, 与 $\bm{k}h=(2,4)$ 处的短而高的涡结构相比。
此外, $\bm{k}h=(0.5,5)$ 处的Taylor时间微尺度在槽道高度上几乎均匀分布, 表明其去相关过程一致。
相反, $\bm{k}h=(2,4)$ 处的流动结构在槽道中心附近表现出较慢的时间相关性衰减, 而在近壁处衰减较快。
可以看出, 白噪声随机力导致Taylor时间微尺度与DNS结果存在较大的预测偏差。
需注意, 理论上使用白噪声强迫的Taylor时间微尺度预期为零~\citep{Wu_2021_PRF}, 但在数值模拟中由于有限的频率范围和分辨率, 它们可能呈现出非常小的值。
与白噪声强迫相比, 学习到的强迫模型显著改善了对所有三个速度分量在整个槽道高度上Taylor时间微尺度的估计。

\begin{figure}
    \centering
    \includegraphics[width=\linewidth]{Img/chap-4/fig13.eps}
    \bicaption{(a) $\tilde{u}$、(b) $\tilde{v}$ 和 (c) $\tilde{w}$ 在波数 (a--c) $\bm{k}h=(0.5,5)$ 和 (d--f) $\bm{k}h=(2,4)$、$Re_\tau=180$ 下的Taylor时间微尺度。
        结果展示了DNS、学习模型预测和白噪声(w.n.)预测。}{Taylor time microscales of (a) $\tilde{u}$, (b) $\tilde{v}$, and (c) $\tilde{w}$ at wavenumber (a--c) $\bm{k}h=(0.5,5)$ and (d--f) $\bm{k}h=(2,4)$ with $Re_\tau=180$.
        Results are shown for the DNS, the learned predictions, and the white-noise (w.n.) predictions.}
    \label{fig: Retau-180-Taylor-two-wave}
\end{figure}

所提出的基于神经算子的强迫模型也能提供积分时间尺度的准确预测。
图~\ref{fig: Retau-180-Integral-two-wave} 为此提供了证据, 该图展示了Reynolds数 $Re_\tau=180$ 的流动在波数 $\bm{k}h=(0.5,5)$ 和 $\bm{k}h=(2,4)$ 处速度分量的积分时间尺度。
积分时间尺度定义为:
\begin{equation}
    \lambda_I(\bm{k},y)=\int_{0}^{\infty}\mid {R}(\bm{k},\tau,y) \mid \dif{\tau} \text{.}
\end{equation}
DNS结果揭示了两种流动结构的相反趋势。
对于 $\bm{k}h=(0.5,5)$ 处的条带结构, 积分时间尺度在近壁处达到最大值, 并向槽道中心逐渐减小, 如图~\ref{fig: Retau-180-Integral-two-wave}(a--c)所示。
相反, $\bm{k}h=(2,4)$ 处的短而高的涡包则呈现出相反的模式, 其积分时间尺度在槽道中心达到峰值, 并向壁面逐渐减小, 如图~\ref{fig: Retau-180-Integral-two-wave}(d--f)所示。
白噪声强迫能产生相似的趋势, 但其预测的时间尺度与DNS结果存在显著偏差, 这表明其难以准确捕捉时间相关函数随时间间隔的衰减。
相比之下, 学习到的强迫模型显著提高了所有三个速度分量在整个槽道高度上积分时间尺度的预测精度, 与DNS结果高度吻合。

\begin{figure}
    \centering
    \includegraphics[width=\linewidth]{Img/chap-4/fig14.eps}
    \bicaption{(a) $\tilde{u}$、(b) $\tilde{v}$ 和 (c) $\tilde{w}$ 在波数 (a--c) $\bm{k}h=(0.5,5)$ 和 (d--f) $\bm{k}h=(2,4)$、$Re_\tau=180$ 下的积分时间尺度。
        结果展示了DNS、学习模型预测和白噪声(w.n.)预测。}{Integral time scales of (a) $\tilde{u}$, (b) $\tilde{v}$, and (c) $\tilde{w}$ at wavenumber (a--c) $\bm{k}h=(0.5,5)$ and (d--f) $\bm{k}h=(2,4)$ for $Re_\tau=180$. Results are shown for the DNS, the learned predictions, and the white-noise (w.n.) predictions.}
    \label{fig: Retau-180-Integral-two-wave}
\end{figure}


\subsection{跨波数与涡黏模型的泛化性能}\label{sec: generalization}
在以上各小节中, 我们仅使用Cess模型对Reynolds数 $Re_\tau=180$ 的流动, 展示了学习强迫模型在波数 $\bm{k}h=(0.5,5)$ 和 $\bm{k}h=(2,4)$ 处的预测能力。
在此, 我们针对不同波数和RANS模型, 评估其在所有三个速度分量的速度频率谱和时间尺度预测方面的性能。
速度频率谱、Taylor时间微尺度及积分时间尺度的预测误差均相对于DNS结果进行评估, 采用下式计算:
\begin{subequations}
    \begin{align}
        \mathcal{E}_{S_{ii}} & =\int_{-\infty}^{\infty}\iint_{-h}^{h} \abs{
            \frac{S_{ii}}{\sbra{\bm{S}}}-
            \frac{S_{ii}^\text{(DNS)}}{\sbra{\bm{S}^\text{(DNS)}}}
        }\delta(y-y') \dif{y}\dif{y'}\dif{\omega}, \label{equ:error-S}      \\
        \mathcal{E}_T        & =\frac{
            \int_{-h}^{h}\abs{\lambda_T-\lambda_T^{\text{(DNS)}}}\dif{y}
        }{
            \int_{-h}^{h}\lambda_T^{\text{(DNS)}}\dif{y}
        },                                                                  \\
        \mathcal{E}_I        & =\frac{
            \int_{-h}^{h}\abs{\lambda_I-\lambda_I^{\text{(DNS)}}}\dif{y}
        }{
            \int_{-h}^{h}\lambda_I^{\text{(DNS)}}\dif{y}
        }.\label{equ: error-lambda}
    \end{align}
\end{subequations}
各速度分量的频率谱均使用相同的缩放因子进行归一化, 即来自所有三个速度分量的总动能 $\sbra{\bm{S}}$, 以确保误差度量在不同分量间具有可比性。
误差度量 $\mathcal{E}_{S_{ii}}$ 表示归一化速度谱的平均差异, 量化了模型预测与DNS结果在频率和壁面法向位置上的偏差。

图~\ref{fig: errors-Retau-180}(a--c)展示了三个速度分量的频率谱 $\langle\tilde{u}_i\tilde{u}_i^\herm\rangle$ 的误差度量。
在所考察的尺度范围内, 学习强迫模型相较于白噪声随机力, 将速度预测误差降低了 $50\%$。
流向速度谱与白噪声强迫的偏差最大, 产生约 $50\%$ 的误差, 而学习强迫的误差则保持在 $20\%$ 左右。
壁面法向和展向速度分量显示出较低的误差水平, 分别约为 $10\%$ 和 $20\%$, 这可能是由于它们的能量强度低于流向分量。
尽管如此, 学习强迫仍能实现进一步的改进, 将预测误差分别降低至 $5\%$ 和 $10\%$。
不同的RANS模型在壁面法向和展向速度上表现出一致的误差水平。
尽管三种RANS模型在流向速度谱上产生的预测误差相对不同, 但它们在所有特征尺度上都显著优于白噪声强迫的预测。
图~\ref{fig: errors-Retau-180}(d--f)展示了各速度分量Taylor时间微尺度 $\lambda_T$ 的误差度量。
与白噪声强迫相比, 使用不同RANS提供涡黏性的学习强迫模型显著降低了预测误差。
如第~\ref{sec: time scale prediction} 节所述, 时域白噪声随机力无法捕捉Taylor时间微尺度 $\lambda_T$, 产生的预测误差大于 $90\%$。
相比之下, 学习到的时域有色强迫将壁面平行速度分量的预测误差降低至约 $20\%$, 在高波数下将壁面法向速度的预测误差降低至 $10\%$ 以下。
此外, 预测精度在三种RANS模型之间保持一致。
图~\ref{fig: errors-Retau-180}(g--i)展示了各速度分量积分时间尺度 $\lambda_I$ 的误差度量。
值得注意的是, 白噪声随机力在积分时间尺度上的表现优于在Taylor时间微尺度上的表现, 尤其是在长条带结构尺度(例如 $|\bm{k}|h \approx 4.5$)下的流向速度。
然而, 在短而高的涡包结构尺度(例如 $|\bm{k}|h \approx 6.7$)下, 白噪声强迫仍导致较高的误差水平。
与白噪声强迫相比, 学习模型在 $\lambda_I$ 上产生的预测误差显著降低。
对于各速度分量及不同RANS模型, 积分时间尺度的预测误差始终低于 $10\%$。
同时, 可以观察到速度谱误差随波数幅值的分布并未与特征时间尺度的误差趋势保持一致。
图~\ref{fig: errors-Retau-180} 中的误差度量表明, 不同RANS模型在与学习强迫模型结合时产生相似的误差水平。
这表明, 无论使用何种RANS模型, 学习模型都能准确预测时间相关函数。

\begin{figure}
    \centering
    \includegraphics[width=\linewidth]{Img/chap-4/fig16.eps}
    \bicaption{(a--c) 速度频率谱误差 $\mathcal{E}_{S_{ii}}$、(d--f) Taylor时间微尺度误差 $\mathcal{E}_T$ 和 (g--i) 积分时间尺度误差 $\mathcal{E}_I$ 在 $Re_\tau=180$ 下不同特征尺度处的误差度量。
    结果展示 (a,d,g) $\tilde{u}$、(b,e,h) $\tilde{v}$ 和 (c,f,i) $\tilde{w}$。
    学习强迫和白噪声(w.n.)强迫的预测误差使用来自DNS(正方形)、Cess(下三角形)、SA(圆形)和 $k$-$\omega$ 模型(上三角形)的涡黏性进行评估。}{Error metrics for (a--c) velocity frequency spectra $\mathcal{E}_{S_{ii}}$, (d--f) Taylor time microscales $\mathcal{E}_T$, and (g--i) integral time scales $\mathcal{E}_I$ with $Re_\tau=180$, at different characteristic scales.
    Results are shown for (a,d,g) $\tilde{u}$, (b,e,h) $\tilde{v}$, and (c,f,i) $\tilde{w}$.
    The prediction errors of the learned and the white-noise (w.n.) forcings are evaluated using eddy viscosities from the DNS (squares), Cess (down triangles), SA (circles), and $k$-$\omega$ models (up triangles).
    }
    \label{fig: errors-Retau-180}
\end{figure}

\subsection{在Reynolds数 $Re_\tau=550$ 下的泛化性能}
最后, 我们展示了学习到的强迫模型在 $Re_\tau=550$ 的槽道流动中, 于图~\ref{fig: characteristic-scales} 所示特征尺度下的泛化能力。
图~\ref{fig: kx-06-kz-06-Retau-550-Cess-vel-psd-curve} 展示了波数 $\bm{k}h=(0.5,5)$ 和 $\bm{k}h=(3,6)$ 处的归一化速度频率谱, 并将学习模型的预测、白噪声预测与DNS结果进行了比较。
谱在流向速度谱的峰值壁面法向位置处评估, 即 $\bm{k}h=(0.5,5)$ 在 $y^+\simeq 397$ 处, $\bm{k}h=(3,6)$ 在 $y^+\simeq 233$ 处, 两者均位于对数律区。
学习预测基于Cess模型, 而白噪声预测使用DNS导出的涡黏性和平均速度。
可以观察到, 学习模型和白噪声预测均正确捕捉了峰值频率。
然而, 白噪声强迫显著低估了谱衰减率, 在远离峰值频率处表现出的谱能量比DNS结果高出几个数量级。
这导致频率带宽变宽。
相比之下, 学习强迫在两个波数下对所有三个速度分量均产生了准确的谱衰减。

\begin{figure}
    \centering
    \includegraphics[width=\linewidth]{Img/chap-4/fig17.eps}
    \bicaption{$Re_\tau=550$ 槽道流动在波数 (a--c) $\bm{k}h=(0.5,5)$ 和 (d--f) $\bm{k}h=(3,6)$ 条件下, 速度分量归一化频率谱 $S_{11}$、$S_{22}$ 和 $S_{33}$ 的对比, 涉及学习模型预测、白噪声(w.n.)预测与DNS结果。
        频率谱展示在峰值壁面法向位置处。
        各图中的黑色虚线表示 $k_xU$ 对应的频率。}{Normalized frequency spectra of the velocity components at wavenumber (a--c) $\bm{k}h=(0.5,5)$ and (d--f) $\bm{k}h=(3,6)$ for channel flows of $Re_\tau=550$ in terms of $S_{11}$, $S_{22}$, and $S_{33}$ with comparison among the learned prediction, the white-noise (w.n.) prediction, and the DNS results.
        The frequency spectra are shown at the peak wall-normal position.
        The black dashed line in each plot indicates the frequency of $k_xU$.}
    \label{fig:  kx-06-kz-06-Retau-550-Cess-vel-psd-curve}
\end{figure}

图~\ref{fig: Retau-550-Taylor-two-wave} 展示了Reynolds数 $Re_\tau=550$ 的槽道流动在波数 $\bm{k}h=(0.5,5)$ 和 $\bm{k}h=(3,6)$ 处的Taylor时间微尺度。
与图~\ref{fig: Retau-180-Taylor-two-wave} 所示的 $Re_\tau=180$ 流动类似, $\bm{k}h=(0.5,5)$ 处的长条带结构展现出相对均匀的Taylor时间微尺度, 且其值大于 $\bm{k}h=(3,6)$ 处短而高的涡包结构。
涡包结构在槽道中心具有最大的Taylor时间微尺度, 并向壁面方向递减。
对于 $\bm{k}h=(0.5,5)$ 处的条带结构, 学习强迫与DNS结果相比也存在明显差异, 但其表现仍显著优于白噪声强迫。
对于 $\bm{k}h=(3,6)$ 处的短而高的涡包结构, 学习强迫准确地预测了Taylor时间微尺度, 与DNS结果高度吻合。

\begin{figure}
    \centering
    \includegraphics[width=\linewidth]{Img/chap-4/fig18.eps}
    \bicaption{(a) $\tilde{u}$、(b) $\tilde{v}$ 和 (c) $\tilde{w}$ 在波数 (a--c) $\bm{k}h=(0.5,5)$ 和 (d--f) $\bm{k}h=(3,6)$、$Re_\tau=550$ 下的Taylor时间微尺度。
        结果展示了DNS、学习模型预测和白噪声(w.n.)预测。}{Taylor time microscales of (a) $\tilde{u}$, (b) $\tilde{v}$, and (c) $\tilde{w}$ at wavenumber (a--c) $\bm{k}h=(0.5,5)$ and (d--f) $\bm{k}h=(3,6)$ with $Re_\tau=550$.
        Results are shown for the DNS, the learned predictions, and the white-noise (w.n.) predictions.}
    \label{fig: Retau-550-Taylor-two-wave}
\end{figure}

图~\ref{fig: Retau-550-Integral-two-wave} 展示了与图~\ref{fig: Retau-550-Taylor-two-wave} 相同波数和Reynolds数下的积分时间尺度。
$\bm{k}h=(0.5,5)$ 处的条带结构表现出比短而高的涡包结构明显更大的积分时间尺度。
与 $Re_\tau=180$ 的情况类似, 短而高的涡包结构在槽道中心具有峰值积分时间尺度, 并向壁面递减。
白噪声强迫导致在两个波数下均与DNS数据存在显著偏差。
相比之下, 学习模型的预测与DNS结果取得了显著一致, 特别是对于 $\bm{k}h=(3,6)$ 处的短而高的涡包结构。

\begin{figure}
    \centering
    \includegraphics[width=\linewidth]{Img/chap-4/fig19.eps}
    \bicaption{$\tilde{u}$、$\tilde{v}$ 和 $\tilde{w}$ 在波数 (a--c) $\bm{k}h=(0.5,5)$ 和 (d--f) $\bm{k}h=(3,6)$、$Re_\tau=550$ 下的积分时间尺度。
        结果展示了DNS、学习模型预测和白噪声(w.n.)预测。}{Integral time scales of $\tilde{u}$, $\tilde{v}$, and $\tilde{w}$ at wavenumber (a--c) $\bm{k}h=(0.5,5)$ and (d--f) $\bm{k}h=(3,6)$ with $Re_\tau=550$.
        Results are shown for the DNS, the learned predictions, and the white-noise (w.n.) predictions.}
    \label{fig: Retau-550-Integral-two-wave}
\end{figure}

图~\ref{fig: errors-Retau-550} 展示了 $Re_\tau=550$ 槽道流动在不同特征尺度下, 各速度分量的频率谱误差 $\mathcal{E}_{S_{ii}}$、Taylor时间微尺度误差 $\mathcal{E}_T$ 和积分时间尺度误差 $\mathcal{E}_I$。
与白噪声强迫相比, 学习强迫将速度谱的预测误差降低了近一半。
具体而言, 流向、壁面法向和展向分量的预测误差分别从($50\%$, $10\%$, $20\%$)降低至($20\%$, $5\%$, $10\%$) 。
对于Taylor时间微尺度, 学习强迫为各速度分量提供的预测误差低于 $20\%$, 较白噪声强迫有显著改进。
同样, 积分时间尺度的预测误差也保持在 $20\%$ 以下。
白噪声强迫在波数 $|\bm{k}|h < 5$ 时对积分时间尺度的预测相对较好, 但在波数 $|\bm{k}|h \approx 6.7$ 处误差接近 $60\%$。
相比之下, 学习强迫模型能在不同波数下提供一致改进的预测。
在图~\ref{fig: errors-Retau-550}(a)中可以观察到, 学习强迫在波数 $\bm{k}h=(0.5,5)$ 处使用 $k$-$\omega$ RANS模型预测的流向速度谱存在稍大的预测误差。
这种差异可能源于 $k$-$\omega$ 模型在Reynolds数 $Re_\tau=550$ 下提供的涡黏性剖面与其他模型存在显著不同, 如附录~\ref{app: RANS} 所示。
学习模型的泛化能力可归因于DeepONet的插值能力, 因为 $Re_\tau=550$ 的涡黏性剖面在很大程度上落在 $Re_\tau=180$ 采样剖面的范围内, 如图~\ref{fig: random-nut-U} 所示。
学习模型的预测性能可在更高Reynolds数(例如 $Re_\tau=5200$)下进一步验证, 但由于该Reynolds数下时间Fourier分析所需存储量极大, 数据获取面临困难。

\begin{figure}
    \centering
    \includegraphics[width=\linewidth]{Img/chap-4/fig20.eps}
    \bicaption{(a--c) 速度频率谱误差 $\mathcal{E}_{S_{ii}}$、(d--f) Taylor时间微尺度误差 $\mathcal{E}_T$ 和 (g--i) 积分时间尺度误差 $\mathcal{E}_I$ 在 $Re_\tau=550$ 下不同特征尺度处的误差度量。
    结果展示 (a,d,g) $\tilde{u}$、(b,e,h) $\tilde{v}$ 和 (c,f,i) $\tilde{w}$。
    学习强迫和白噪声(w.n.)强迫的预测误差使用来自DNS、Cess、SA 和 $k$-$\omega$ 模型的涡黏性进行评估。}{
    Error metrics for (a--c) velocity frequency spectra $\mathcal{E}_{S_{ii}}$, (d--f) Taylor time microscales $\mathcal{E}_T$, and (g--i) integral time scales $\mathcal{E}_I$ with $Re_\tau=550$, evaluated at different characteristic scales.
    Results are shown for (a,d,g) $\tilde{u}$, (b,e,h) $\tilde{w}$, and (c,f,i) $\tilde{w}$.
    The prediction errors of the learned and the white noise (w.n.) forcings are evaluated using eddy viscosities from the DNS, Cess, SA, and $k$-$\omega$ models.}
    \label{fig: errors-Retau-550}
\end{figure}

\section{结论}\label{sec: conclusion}
在本研究中, 我们提出使用基于神经算子的时域有色强迫, 基于RANS预测的涡黏性和平均速度来预测大尺度流动结构的时空特性。
这是通过结合涡黏性强化预解算子与基于神经算子的随机力模型实现的。
具体而言, 基于DeepONet架构的神经算子用于构建从涡黏性和平均速度到时域有色随机力谱的映射。
进一步地, 预解算子用于将获得的随机力谱映射到时空速度统计量。
通过这种方式, 神经算子增强的预解模型能够准确预测速度统计量, 而无需依赖现有数据, 这使得学习到的随机力模型能够应用于不同的流动条件。
此外, 基于DeepONet的模型能够灵活评估随机力的任意定义域位置, 从而允许预测不同时空尺度下的速度谱。
同时, 基于神经算子的强迫模型能够很好地泛化到不同RANS提供的涡黏性和平均速度剖面。

基于神经算子的随机力首先基于 $Re_\tau=180$ 槽道流动的直接数值模拟(DNS)数据训练。
我们展示了学习到的神经算子模型能够准确预测随机力谱和大尺度流动结构的速度响应。
特别是, 与传统的白噪声强迫相比, 学习模型能够显著改进时间相关性及特征时间尺度的预测。
我们还展示了基于神经算子的模型能够与不同的RANS模型结合, 为速度统计量提供一致的预测。
此外, 所提出的神经算子可应用于 $Re_\tau=550$ 的槽道流动, 对大尺度流动结构的时空特性做出准确预测。

在当前工作中, 强迫模型是针对槽道流动的大尺度结构学习的, 在应用于预测小尺度流动统计量时可能存在局限。
然而, 我们注意到, 所提出的方法可用于训练针对小尺度流动结构的特定神经算子模型, 并采用基于内尺度的归一化。
此外, 当前工作中的学习模型难以泛化到槽道流动之外的不同构型。
针对其他流动场景(例如射流)的特定随机力模型, 可使用基于神经算子的方法从高保真数据中学习。
另外, 本研究忽略了随机力的壁面法向相关性和分量间相关性, 这些可在未来的工作中进一步纳入神经算子模型。
此外, 本研究聚焦于涡黏性强化预解框架中的随机力模型, 该模型通过随机力补偿了RANS模型的误差。
值得进一步研究的是, 如何利用数据驱动方法同时改进RANS模型和非线性强迫模型, 特别是在具有显著RANS模型误差的复杂流动(例如分离流)中。
如此, RANS模型能够提供合适的涡黏性和准确的平均速度, 从而改进基于预解的线性算子。
强迫模型进一步产生时域有色的输入激励, 最终获得准确的速度谱。